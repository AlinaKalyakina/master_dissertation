\section{Обзор существующих решений рассматриваемой зада­чи или ее модификаций}
Рассмотрим существующие методы обнаружения инсайдерских угроз, сделаем их сравнение. Обратим особое внимание на использование контентной составляющей поведения пользователей, сравним используемые в различных работах подходы. В конце, рассмотрим существующие для наддной работы наборы данных.
\subsection{Обзор существующих подходов}
Задача обнаружения инсайдерских угроз может быть рассмотрена с двух ракурсов: как задача поиска аномалий, и как задача обучения с учителем.
В первом случае...
Сравнение работ по критериям:
\begin{enumerate}
\item Supervised/unsupervised
\item Количество ''этапов''
\item Использование контекста
\item Использование контента
\item Использованные наборы данных
\item Качество на CERT
\end{enumerate}
Сравнение самих моделей??? (это есть в ревью, но не факт, что полезно)
\begin{enumerate}
\item Supervised/unsupervised
\item Использование контекста
\item Использование контента
\item Использованные наборы данных
\item Качество на CERT
\end{enumerate}
\subsection{Подходы к работе с контентом}
ABSA - анализ, LDA, эмбеддинги. Как это работает примеры работ с этим.
\subsection{Обзор наборов данных}
\clearpage