\section{Обзор существующих решений рассматриваемой зада­чи или ее модификаций}
Рассмотрим существующие подходы к задаче обнаружения инсайдерских угроз, сделаем их сравнение. Как уже было описано ранее, поведенческие данные можно разделить на два вида: контент --- неструктурированные данные и контекст --- структурированные. В литуратуре можно встретить подходы использующие только контент(\cite{rules}, \cite{absa}), только контекст(\cite{lac}, \cite{granual}, \cite{lstm_cnn}, \cite{cnn_lstm}, \cite{gru}) и использующие оба вида(\cite{anomalyalgo}, \cite{suites}) поведенческих данных.

В первом разделе рассмотрим предлагаемые в литературе подходы к решению задачи обнаружения инсайдеров. Затем обратим особое внимание на способы использования в работах контентной составляющей поведенческих данных, сравним используемые в различных работах подходы. В конце, рассмотрим существующие для данной работы наборы данных.
\subsection{Обзор существующих подходов}

Базовый подход к любой задаче обнаружения чего-либо - придумать правила. Так поступили авторы \cite{rules}. В этой работе была предложена иерархия инсайдеров и определены присущие им черты характера. Затем по контенту пользовательского поведения с помощью сервиса IBM Personality Insights были определены черты характера пользователей, которые классифицировались по предложенным правилам. 

Задача обнаружения инсайдерских угроз может быть рассмотрена с двух сторон: как задача обнаружения аномалий, и как задача обучения с учителем. Оба имеют недостатки: 
\begin{itemize}
\item Не каждое аномальное поведение представляет собой угрозу. 
\item Для обучения алгоритма с учителем необходима разметка. На данный момент есть проблема с размеченными наборами данных. 
\end{itemize}

\subsubsection{Обнаружение инсайдеров как задача поиска аномалий}

В \cite{lac} авторы для каждого сотрудника обучили LSTM-автокодировщик его ``нормальному'' поведению, а также построили ``граф''  сообщества пользователей, в котором ребра - общение с помощью писем . Затем они разделили всех пользователей на непересекающиеся сообщества Лувенским методом\cite{louvain} и посчитали среднюю ошибку реконструкции для каждого пользователя на всех обученных маделях его группы. Чем больше эта ошибка, тем более аномально поведение пользователя.

В \cite{anomalyalgo} авторы моделируют поведение каждого пользователя с трех сторон:
\begin{itemize}
\item Аггрегация дневной активности - контекст пользовательского контекста за каждый день
\item LDA-моделирование контента электронных писем
\item Положение пользователя в грае коммуникации внутри компании
\end{itemize}
Затем применяют четыре алгоритма детекции аномалий:
\begin{itemize}
\item Метод главных компонент. В качестве значения аномалий использовалась ошибка реконструкции.
\item Метод К ближайших соседей.
\item Оценка параметров нормального распределения на тренировочной выборке и последующая оценка вероятности того, что новые наблюдения принадлежат оцененному распределению.
\item Оценка плотности распределения окном Парзена и последующая оценка вероятности принадлежности новых наблюдений данному.
\end{itemize}

В работе \cite{absa} применяется аспектно-ориентированный анализ эмоциональной окраски (англ. ABSA --- Aspect-based sentiment analysis) на контентных поведенческих данных. В качетсве ABSA-модели использовалась современная нейросетевая рекуррентная модель с механизмом внимания. Для обнаружения аномалий используется Изолирующий лес.

\subsubsection{Обнаружение инсайдеров как задача обучения с учителем}

В \cite{suites} авторы пробуют применить порядка сорока методов классификации и приходят к выводу, что лучше всего в этой задаче себя показывает случайный лес. В качестве признакового пространства в работе использовались эмоциональный факторы из писем и контекстные данные. Инетересно, что пользоватлеи классифицировались сразу на пять классов: добропорядочный, бывший работник, вор, тот, кто сливает информацию, и саботажник.

В \cite{granual} авторы пробуют применить сразу четыре алгоритма обучения с учителем: Логистическую регрессию, Случайный лес, Нейронную сеть(архитектура не показана) и XGBoost. В работе используются только контекстные поведенческие данный, которые агрегируются на разных масштабах: от недели до N действий пользователя в течение сессии. Авторы исследуют эффективность агрегации контекста пользователя на разных промежутках времени и приходят к выводу, что наиболее эффективно аггрегировать поведенческий контекст в течение сессии или рабочего дня.

Поведение пользователя представляет собою последовательность его действий, напоминающую текст. Поэтому во многих работах применяются хорошо показавшие себя на задачах обработки естественного языка (англ. NLP - Natural Language Processing) идеи: использование рекуррентных нейросетей, использование механизма внимания (англ. Attention).  

В \cite{lstm_cnn} предлагается двухэтапный подход: сначала авторы обучают LSTM-сеть поведению пользователей, а затем извлекают из нее признаки и подают их на вход сверточной сети-классификатору.
В \cite{cnn_lstm} авторы применяют обратный подход: сначала сверточной сетью извлекаются признаки, а затем они подаются на вход рекуррентному классификатору с LSTM-ячейками. 

Еще один нейросетевой подход описывается в \cite{gru}. В работе авторы предлагают напрямую применять GRU-сеть для обнаружения инсайдеров.

В работе \cite{attention} авторы исследовали применение механизма внимания в рекуррентной сети с LSTM-ячейками для обнаружения инсайдерских угроз. 

\subsubsection{Подходы к работе с контекстом}
One-hot, SMOTE\cite{cnn_lstm}, агрегация

\subsubsection{Подходы к работе с контентом}
ABSA - анализ, LDA, эмбеддинги. Как это работает примеры работ с этим.

\subsubsection{Классификаторы}

\subsubsection{Сравнение работ}
Сравнение работ по критериям:
\begin{enumerate}
\item Supervised/unsupervised
\item Что есть признаки
\item Использование контекста
\item Использование контента
\item Использованные наборы данных
\item Качество на CERT
\end{enumerate}

\subsection{Обзор наборов данных}
\label{sec:sample}
\clearpage
