\section*{Введение}
\textbf{Инсайдер} - человек, который в силу своего служебного положения или иных обстоятельств имеет доступ к конфиденциальной информации внутри компании.
\textbf{Инсайдерская угроза} - это вредоносные для комании угроза, исходящая от инсайдера. 

Инсайдеров и инсайдерские атаки можно классифицировать по: 
\begin{itemize}
\item \textbf{Типу доступа}: по сети или физическому. По своей природе инсайдер имеют авторизованный доступ к сети и/или физический доступ к данным компании. Исследования показывают, что 2/3 инцидентов совершены через сеть.\cite{review2020}
\item \textbf{Стратегии инсайдера}: предатель, притворщик и непреднамеренный инсайдер. Предатели - пользователи, которые находятся внутри организации и умышленно злоупотрябляют своими правами. Притворщики - внешние злоумышленники, которые используют украденные идентификационные данные чтобы выдать себя за инсайдера. Непреднамеренный инсайдер - текущий сотрудник, который без злого умысла причинаяет вред или увеличивает уязвимость компании в будущем. Исследования показывают, что 92\% исайдерских атак осуществляются именно предателями.\cite{review2020}
\item \textbf{Виду атаки}: саботаж, кража (интеллектуальной собственности), мошенничество и шпионаж. Саботаж - нанесение вреда организации, например, заведомое создание ``бекдоров'' , под мошенничеством чаще вывод средств организации обманным путем. Самый частый вид атаки - саботаж\cite{review2020}, обычно его совершают недовольные работники.
\end{itemize}

Ущерб от инсайдерских атак растет с каждым годом, так по данным Ponemon\cite{ponemon} суммарный ущерб от подобных атак вырос на 31\% с \$8.76 млн в 2018 году до \$11.45 млн в 2020 году. Кроме того на 47\% увеличилось количество инцидентов (с 3,200 в 2018 до 4,716 в 2020).  По данным 

Для снижения ущерба от инсайдерских атак возможно компании используют следующие технологии и практики (перечислены пять наиболее эффективных по данным\cite{ponemon} в порядке убывания эффективности):
\begin{enumerate}
\item \textbf{UEBA-системы (англ. User and Entity behavior anaytics)} анализаруют поведение пользователей. Более подробно рассмотрим их в дальнейшем.
\item \textbf{Системы управления доступом (PAM,  англ. Privileged access management)}. Доступ к данным предоставляется только непосредственно использующих их сотрудникам, что уменьшает риск утечек.
\item \textbf{Обучение сотрудников}. Многие сотрудники совершают инсайдерские атаки не преднамеренно: редко меняют пароль, случайно разглашают данные, постоянное обучение помогает предотвратить такие инциденты. По данным Ponemon это самый часто применяемый метод, хотя и не самый эффективный. 
\item \textbf{SIEM-системы (англ. Security incident\&event)} анализируют в реальном времени события безопасности.
\item \textbf{Обмен разведданными об угорозах}.
\end{enumerate}
При составлении приведенного выше рейтинга Ponemon учитывали затраты на внездрение практики и ее поддержку.
Интересно, {DLP-cистемы (англ. Data loss prevention)} вторые по популярности по данныи Ponemon занимают последнее место в рейтинге эффективности. DLP-cистемы следят за пересекающими периметр информационный системы потоками данных.

Основное преимущество UEBA-систем -- способность обнаруживать ранние признаки готовящейся атаки, в то время как остальные системы пытаются предотвратить непосредственно саму атаку и поэтому ``не имеют права на ошибку''. Изучим рынок коммерческих UEBA-систем подробнее. Любые UEBA-системы описараются на три столба\cite{gartner}:
\begin{itemize}
\item \textbf{Случаи использования}. Разработчики должны предусмотреть возможные инциденты, например, скомпроментированного пользователя, предателя.
\item \textbf{Используемые данные}. Система может анализировать такие данные, как логи, сетевой трафик, используемый пользователями контент
\item \textbf{Используемые методы аналитики}. По исследованию\cite{gartner} на данный момент системы работают на
\end{itemize}

\clearpage